
\documentclass[preprint,authoryear]{sigplanconf}

% The following \documentclass options may be useful:

% preprint      Remove this option only once the paper is in final form.
% 10pt          To set in 10-point type instead of 9-point.
% 11pt          To set in 11-point type instead of 9-point.
% authoryear    To obtain author/year citation style instead of numeric.

\usepackage{lipsum}
\usepackage{amsmath}
\usepackage{proof}

\def\turn{\vdash}
\def\conv{\approx}

\newcommand{\fun}[1]{\textmd{#1}}
\newcommand{\cir}[1]{\textcircled{\raisebox{-0.9pt}{#1}}}

\begin{document}

\special{papersize=8.5in,11in}
\setlength{\pdfpageheight}{\paperheight}
\setlength{\pdfpagewidth}{\paperwidth}

\conferenceinfo{CONF 'yy}{Month d--d, 20yy, City, ST, Country} 
\copyrightyear{20yy} 
\copyrightdata{978-1-nnnn-nnnn-n/yy/mm} 
\doi{nnnnnnn.nnnnnnn}

% Uncomment one of the following two, if you are not going for the 
% traditional copyright transfer agreement.

%\exclusivelicense                % ACM gets exclusive license to publish, 
                                  % you retain copyright

%\permissiontopublish             % ACM gets nonexclusive license to publish
                                  % (paid open-access papers, 
                                  % short abstracts)

\titlebanner{DRAFT}        % These are ignored unless
%% \preprintfooter{short description of paper}   % 'preprint' option specified.

\title{Expressionless Weak-Head Normal Forms}
%% \subtitle{Subtitle Text, if any}

\authorinfo{Larry Diehl}
           {Portland State University}
           {ldiehl@cs.pdx.edu}

\authorinfo{Tim Sheard}
           {Portland State University}
           {sheard@cs.pdx.edu}

\maketitle

\begin{abstract}
\lipsum[2-3]
\end{abstract}

\category{CR-number}{subcategory}{third-level}

% general terms are not compulsory anymore, 
% you may leave them out
\terms
term1, term2

\keywords
keyword1, keyword2

\section{Introduction}

The text of the paper begins here.

Lots of text.

More text.

Lots of text.

More text.


Lots of text.

More text.

Lots of text.

More text.

\section{Dependent Type Checking}

The motivation for weak-head normal forms in dependent type theory is
to strike a balance between simplicity and efficiency in the
implementation of the type checker. 

\subsection{Specification}

Before considering the {\it implementation} of a dependent type checker,
let's consider the complexities that arise in the
{\it specification} of dependent typing rules.
The rule for dependent application exposes the complexities
that arise in dependent type checking.

$$
\infer[]
  {\Gamma \turn f~a : B[a]}
{
  \Gamma \turn f : \Pi~A~B
  &
  \Gamma \turn a : A
}
$$

There are 3 interesting
constraints in this rule, and all of them have to do with what is
happening in the {\it type} position of the typing judgement.

\begin{enumerate}
\item The type of the function $f$ must be $\Pi~A~B$, thus it must at
  least be in head-normal form.
\item The type of $a$ must be equal to the domain of $f$.
\item The type of the application $f a$ is the substitution of $a$ for
  the bound variable in $B$.
\end{enumerate}

Because redexes can appear in dependent types, it is sometimes
necessary for types to be reduced before a typing rule becomes valid.
Otherwise, the following 3 problems can violate the aforementioned
constraints.

\begin{enumerate}
\item The type of $f$ is a redex like $(\lambda x. x)~(\Pi~A~B)$.
\item The type of $a$ is $A$, while the codomain of $f$ is a redex like
$(\lambda x. x)~A$.
\item Substituting the term $a$ into $B$ can create a redex, resulting
  in problems 1 and 2 somewhere else in type checking.
\end{enumerate}

The specification of typing rules can sweep all of these
problems under the rug by via a conversion rule.

$$
\infer[]
  {\Gamma \turn a : B}
{
  \Gamma \turn a : A
  &
  A \conv B
}
$$

The conversion relation $\conv$ considers types to equal after
reducing redexes. Hence, $\conv$ at least implements $\alpha$-$\beta$
equality, but it may implement additional forms of equality like $\eta$.

\subsection{Implementation}

Dependent type theory can be specified succinctly because the
conversion relation is nondeterministic. In contrast, an
implementation of a dependent type checker in a functional language
must be deterministic. Additionally, the implementation must be
reasonably efficient to execute. 

Below we consider 4 different implementations of type checking
function application. The 4 implementations differ in the syntactic
grammar used for types: First expressions, then normal forms, then
weak-head normal forms with expression closures, and finally our novel
weak-head normal forms with expressionless closures.

\paragraph{Expression Types}

Consider the function case of a type checker for expressions below.
For simplicity, we assume that all terms are annotated enough for us
to infer types instead of checking them. Additionally, the terms are
in de Bruijn notation so we do not need to deal with $\alpha$-renaming.
The input of \texttt{infer} is an expression
value, and the output is an expression type in a type checking monad.

\begin{verbatim}
infer :: Exp -> TCM Exp
infer (App f a) = do
  Pi _A _B <- infer f
  _A'      <- infer a
  unless (_A == _A') $
    throwError "Domain not convertible to argument"
  return . norm (B `sub` a)
\end{verbatim}

Whether or not the code above is correct depends on the rest of the
implementation of \texttt{infer}, what the conversion function
\texttt{(==)} does, and what the normalization function
\texttt{norm} does. Because the grammar of expressions is so flexible,
we can never be sure if a given expression is in normal form,
weak-head normal form, or if the head position is a redex.

The easiest way to get the implementation correct is to make
\texttt{norm} compute to full normal form. This way,
conversion \texttt{(==)} can be a derived syntactic equality. The
(albeit more complex) implementation makes \texttt{norm} compute only
to weak-head normal form, and conversion \texttt{(==)} first check
syntactic equality, and then compute away redexes if syntactic
equality fails. Whatever normal form we compute to, we need to make
sure that all cases of \texttt{infer} do it so that we can
successfully match against \texttt{Pi \_A \_B}.

Why is it a good idea to prefer weak-head normal forms over normal
forms? The normal forms can get quite large! Consider type checking
the application case where the type of the argument (\texttt{\_A'})
and the domain (\texttt{\_A}) are the finite set type
\texttt{Fin (2 ** 1024)}. Computing both of these types to normal form
and comparing them syntactially is not practical.

\subsection{Independence}

\appendix
\section{Appendix Title}

This is the text of the appendix, if you need one.

\acks

Acknowledgments, if needed.

% We recommend abbrvnat bibliography style.

\bibliographystyle{abbrvnat}

% The bibliography should be embedded for final submission.

\begin{thebibliography}{}
\softraggedright

\bibitem[Smith et~al.(2009)Smith, Jones]{smith02}
P. Q. Smith, and X. Y. Jones. ...reference text...

\end{thebibliography}


\end{document}

%                       Revision History
%                       -------- -------
%  Date         Person  Ver.    Change
%  ----         ------  ----    ------

%  2013.06.29   TU      0.1--4  comments on permission/copyright notices

