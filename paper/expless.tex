
\documentclass[preprint,authoryear]{sigplanconf}

% The following \documentclass options may be useful:

% preprint      Remove this option only once the paper is in final form.
% 10pt          To set in 10-point type instead of 9-point.
% 11pt          To set in 11-point type instead of 9-point.
% authoryear    To obtain author/year citation style instead of numeric.

\usepackage{lipsum}
\usepackage{amsmath}
\usepackage{proof}

\def\turn{\vdash}
\def\conv{\approx}

\newcommand{\fun}[1]{\textmd{#1}}

\begin{document}

\special{papersize=8.5in,11in}
\setlength{\pdfpageheight}{\paperheight}
\setlength{\pdfpagewidth}{\paperwidth}

\conferenceinfo{CONF 'yy}{Month d--d, 20yy, City, ST, Country} 
\copyrightyear{20yy} 
\copyrightdata{978-1-nnnn-nnnn-n/yy/mm} 
\doi{nnnnnnn.nnnnnnn}

% Uncomment one of the following two, if you are not going for the 
% traditional copyright transfer agreement.

%\exclusivelicense                % ACM gets exclusive license to publish, 
                                  % you retain copyright

%\permissiontopublish             % ACM gets nonexclusive license to publish
                                  % (paid open-access papers, 
                                  % short abstracts)

\titlebanner{DRAFT}        % These are ignored unless
%% \preprintfooter{short description of paper}   % 'preprint' option specified.

\title{Expressionless Weak-Head Normal Forms}
%% \subtitle{Subtitle Text, if any}

\authorinfo{Larry Diehl}
           {Portland State University}
           {ldiehl@cs.pdx.edu}

\authorinfo{Tim Sheard}
           {Portland State University}
           {sheard@cs.pdx.edu}

\maketitle

\begin{abstract}
\lipsum[2-3]
\end{abstract}

\category{CR-number}{subcategory}{third-level}

% general terms are not compulsory anymore, 
% you may leave them out
\terms
term1, term2

\keywords
keyword1, keyword2

\section{Introduction}

The text of the paper begins here.

Lots of text.

More text.

Lots of text.

More text.


Lots of text.

More text.

Lots of text.

More text.

\section{Dependent Type Checking}

The motivation for weak-head normal forms in dependent type theory is
to strike a balance between simplicity and efficiency in the
implementation of the type checker. 

\subsection{Specification}

Before considering the {\it implementation} of a dependent type checker,
let's consider the complexities that arise in the
{\it specification} of dependent typing rules.
The rule for dependent application exposes the complexities
that arise in dependent type checking.

$$
\infer[]
  {\Gamma \turn f~a : B[a]}
{
  \Gamma \turn f : \Pi~A~B
  &
  \Gamma \turn a : A
}
$$

There are 3 interesting
constraints in this rule, and all of them have to do with what is
happening in the {\it type} position of the typing judgement.

\begin{enumerate}
\item The type of the function $f$ must be $\Pi~A~B$, thus it must at
  least be in head-normal form.
\item The type of $a$ must be equal to the domain of $f$.
\item The type of the application $f a$ is the substitution of $a$ for
  the bound variable in $B$.
\end{enumerate}

Because redexes can appear in dependent types, it is sometimes
necessary for types to be reduced before a typing rule becomes valid.
Otherwise, the following 3 problems can violate the aforementioned
constraints.

\begin{enumerate}
\item The type of $f$ is a redex like $(\lambda x. x)~(\Pi~A~B)$.
\item The type of $a$ is $A$, while the codomain of $f$ is a redex like
$(\lambda x. x)~A$.
\item Substituting the term $a$ into $B$ can create a redex, resulting
  in problems 1 and 2 somewhere else in type checking.
\end{enumerate}

The specification of typing rules can sweep all of these
problems under the rug by via a conversion rule.

$$
\infer[]
  {\Gamma \turn a : B}
{
  \Gamma \turn a : A
  &
  A \conv B
}
$$

The conversion relation $\conv$ considers types to equal after
reducing redexes. Hence, $\conv$ at least implements $\alpha$-$\beta$
equality, but it may implement additional forms of equality like $\eta$.

\subsection{Implementation}



\appendix
\section{Appendix Title}

This is the text of the appendix, if you need one.

\acks

Acknowledgments, if needed.

% We recommend abbrvnat bibliography style.

\bibliographystyle{abbrvnat}

% The bibliography should be embedded for final submission.

\begin{thebibliography}{}
\softraggedright

\bibitem[Smith et~al.(2009)Smith, Jones]{smith02}
P. Q. Smith, and X. Y. Jones. ...reference text...

\end{thebibliography}


\end{document}

%                       Revision History
%                       -------- -------
%  Date         Person  Ver.    Change
%  ----         ------  ----    ------

%  2013.06.29   TU      0.1--4  comments on permission/copyright notices

